\documentclass[12pt,]{article}
\usepackage[left=1in,top=1in,right=1in,bottom=1in]{geometry}
\newcommand*{\authorfont}{\fontfamily{phv}\selectfont}
\usepackage[]{mathpazo}


  \usepackage[T1]{fontenc}
  \usepackage[utf8]{inputenc}




\usepackage{abstract}
\renewcommand{\abstractname}{}    % clear the title
\renewcommand{\absnamepos}{empty} % originally center

\renewenvironment{abstract}
 {{%
    \setlength{\leftmargin}{0mm}
    \setlength{\rightmargin}{\leftmargin}%
  }%
  \relax}
 {\endlist}

\makeatletter
\def\@maketitle{%
  \newpage
%  \null
%  \vskip 2em%
%  \begin{center}%
  \let \footnote \thanks
    {\fontsize{18}{20}\selectfont\raggedright  \setlength{\parindent}{0pt} \@title \par}%
}
%\fi
\makeatother




\setcounter{secnumdepth}{0}




\title{Identifying Opinion Change and Knowledge Levels of Deliberative Groups
(Natural Language Processing) \thanks{Code and data available at: github.com/rossdahlke/cs\_230\_project}  }



\author{\Large Ross Dahlke
(\href{mailto:rdahlke@stanford.edu}{\nolinkurl{rdahlke@stanford.edu}})\vspace{0.05in} \newline\normalsize\emph{}  }


\date{}

\usepackage{titlesec}

\titleformat*{\section}{\normalsize\bfseries}
\titleformat*{\subsection}{\normalsize\itshape}
\titleformat*{\subsubsection}{\normalsize\itshape}
\titleformat*{\paragraph}{\normalsize\itshape}
\titleformat*{\subparagraph}{\normalsize\itshape}





\newtheorem{hypothesis}{Hypothesis}
\usepackage{setspace}


% set default figure placement to htbp
\makeatletter
\def\fps@figure{htbp}
\makeatother


% move the hyperref stuff down here, after header-includes, to allow for - \usepackage{hyperref}

\makeatletter
\@ifpackageloaded{hyperref}{}{%
\ifxetex
  \PassOptionsToPackage{hyphens}{url}\usepackage[setpagesize=false, % page size defined by xetex
              unicode=false, % unicode breaks when used with xetex
              xetex]{hyperref}
\else
  \PassOptionsToPackage{hyphens}{url}\usepackage[draft,unicode=true]{hyperref}
\fi
}

\@ifpackageloaded{color}{
    \PassOptionsToPackage{usenames,dvipsnames}{color}
}{%
    \usepackage[usenames,dvipsnames]{color}
}
\makeatother
\hypersetup{breaklinks=true,
            bookmarks=true,
            pdfauthor={Ross Dahlke
(\href{mailto:rdahlke@stanford.edu}{\nolinkurl{rdahlke@stanford.edu}}) ()},
             pdfkeywords = {},  
            pdftitle={Identifying Opinion Change and Knowledge Levels of Deliberative Groups
(Natural Language Processing)},
            colorlinks=true,
            citecolor=blue,
            urlcolor=blue,
            linkcolor=magenta,
            pdfborder={0 0 0}}
\urlstyle{same}  % don't use monospace font for urls

% Add an option for endnotes. -----


% add tightlist ----------
\providecommand{\tightlist}{%
\setlength{\itemsep}{0pt}\setlength{\parskip}{0pt}}

% add some other packages ----------

% \usepackage{multicol}
% This should regulate where figures float
% See: https://tex.stackexchange.com/questions/2275/keeping-tables-figures-close-to-where-they-are-mentioned
\usepackage[section]{placeins}


\begin{document}
	
% \pagenumbering{arabic}% resets `page` counter to 1 
%
% \maketitle

{% \usefont{T1}{pnc}{m}{n}
\setlength{\parindent}{0pt}
\thispagestyle{plain}
{\fontsize{18}{20}\selectfont\raggedright 
\maketitle  % title \par  

}

{
   \vskip 13.5pt\relax \normalsize\fontsize{11}{12} 
\textbf{\authorfont Ross Dahlke
(\href{mailto:rdahlke@stanford.edu}{\nolinkurl{rdahlke@stanford.edu}})} \hskip 15pt \emph{\small }   

}

}






\vskip -8.5pt


 % removetitleabstract

\noindent \doublespacing 

\hypertarget{introduction}{%
\section{1. Introduction}\label{introduction}}

Deliberative Polling asks, ``What do people think if they are given
information and circumstances to the discuss policy issues?'' (Stanford
University Center for Deliberative Democracy, n.d.) Stanford Professor
Jim Fishkin and the Center for Deliberative Democracy (CDD) conduct
Deliberative Polls where people come together for a weekend where they
are polled on their opinions, spend the weekend deliberating in groups,
and are polled afterwards about their opinions. Fishkin finds that
people change their opinions after deliberations.

In this project, I will combine transcripts and survey results from
Deliberative Polls to answer:

\textbf{Can we build an NLP model that predicts whether a group will
change their opinions?}

If time permits, I hope to deploy this model to predict opinion change
on some deliberative group such as a town hall.

This project was inspired by previous research on attitude change in the
``Change my View'' subreddit (Priniski and Horne 2018).

\hypertarget{datasetfeaturespreprocessing}{%
\section{2.
Dataset/Features/Preprocessing}\label{datasetfeaturespreprocessing}}

Although there have been dozens of Deliberative Polls conducted across
the world, many are not in English, have not been translated into
English, or do not have high-quality English transcriptions. I requested
all datasets that have high-quality English transcriptions from the CDD.
I am anticipating at least three datasets from:

\begin{enumerate}
\def\labelenumi{\arabic{enumi}.}
\tightlist
\item
  Ghana
\item
  Stanford Undergraduates
\item
  American High Schoolers
\end{enumerate}

In each of these datasets there are transcripts from about 20 groups and
their survey responses. In total, I anticipate having about 60 examples
of transcripts and survey results before and after deliberation. Of
these 60 examples, I plan to use 51 as a training set, 6 as a test set,
and 3 as a validation set.

Preprocessing will include all of the standard NLP preprocessing such as
padding, truncation, and tokenization. Since I will be primarily using
transformers, I will experiment whether text standardization such as
punctuation, capitalization, and stop word removal improves the accuracy
of the models.

\hypertarget{challenges}{%
\section{3. Challenges}\label{challenges}}

The biggest challenges of this project will be dealing with the small
number of training examples. I will use pre-trained transformers to try
to alleviate this issue.

\hypertarget{methods}{%
\section{4. Methods}\label{methods}}

I will first try LSTM on the dataset to establish a baseline. However,
given the small dataset that I'll be using, I primarily use
transformers. I will test multiple models, including:

\begin{itemize}
\tightlist
\item
  XLNet
\item
  BERT
\item
  GPT-2
\item
  CTRL
\item
  Longformer
\end{itemize}

My hypothesis is that the detection of opinion change will not be based
\textbf{how} people talk instead of \textbf{what} they talk about. I
will experiment with different methods to capture this change in
\textbf{how} people talk by adding in linguistic features. Early work
has shown that incorporating sentence dependency features into one's
model can improve performance (Komninos and Manandhar 2016), but I hope
to extend this work by incorporating sentence dependencies and other
linguistic features such as average length of uninterrupted talking.

\hypertarget{evaluation}{%
\section{5. Evaluation}\label{evaluation}}

Quantitatively, I will evaluate on a discrete label of whether the group
changed their opinion or not by using a confusion matrix to calculate
recall, precision, and overall accuracy. Qualitatively, I will examine
training data that is incorrectly classified to see if there are
specific problems with that example such as anomalous lengths or
deliberation content.

\hypertarget{references}{%
\section*{References}\label{references}}
\addcontentsline{toc}{section}{References}

\hypertarget{refs}{}
\leavevmode\hypertarget{ref-komninos}{}%
Komninos, Alexandros, and Suresh Manandhar. 2016. ``Dependency Based
Embedding for Sentence Classification Tasks.'' \emph{Proceedings of
NAACL-HLT}, June, 1490--1500.

\leavevmode\hypertarget{ref-cmv}{}%
Priniski, John Hunter, and Zach Horne. 2018. ``Attitude Change on
Reddit's Change My View.'' \emph{Conference: Proceedings of the 40th
Annual Meeting of the Cognitiee Science Society, Madison, Wisconsin}.
\url{https://www.researchgate.net/publication/333175400_Attitude_Change_on_Reddit's_Change_My_View}.

\leavevmode\hypertarget{ref-cdd}{}%
Stanford University Center for Deliberative Democracy. n.d. ``What Is
Deliberative Polling®?'' \emph{CDD}.
\url{https://cdd.stanford.edu/what-is-deliberative-polling/}.





\newpage
\singlespacing 
\end{document}
