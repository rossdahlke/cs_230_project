\documentclass[11pt,]{article}
\usepackage[left=1in,top=1in,right=1in,bottom=1in]{geometry}
\newcommand*{\authorfont}{\fontfamily{phv}\selectfont}
\usepackage[]{mathpazo}


  \usepackage[T1]{fontenc}
  \usepackage[utf8]{inputenc}




\usepackage{abstract}
\renewcommand{\abstractname}{}    % clear the title
\renewcommand{\absnamepos}{empty} % originally center

\renewenvironment{abstract}
 {{%
    \setlength{\leftmargin}{0mm}
    \setlength{\rightmargin}{\leftmargin}%
  }%
  \relax}
 {\endlist}

\makeatletter
\def\@maketitle{%
  \newpage
%  \null
%  \vskip 2em%
%  \begin{center}%
  \let \footnote \thanks
    {\fontsize{18}{20}\selectfont\raggedright  \setlength{\parindent}{0pt} \@title \par}%
}
%\fi
\makeatother




\setcounter{secnumdepth}{0}




\title{Predicting Opinion Change in Deliberative Groups (Natural Language
Processing) \thanks{Code and data available at: github.com/rossdahlke/cs\_230\_project}  }



\author{\Large Ross Dahlke
(\href{mailto:rdahlke@stanford.edu}{\nolinkurl{rdahlke@stanford.edu}})\vspace{0.05in} \newline\normalsize\emph{}  }


\date{}

\usepackage{titlesec}

\titleformat*{\section}{\normalsize\bfseries}
\titleformat*{\subsection}{\normalsize\itshape}
\titleformat*{\subsubsection}{\normalsize\itshape}
\titleformat*{\paragraph}{\normalsize\itshape}
\titleformat*{\subparagraph}{\normalsize\itshape}





\newtheorem{hypothesis}{Hypothesis}
\usepackage{setspace}


% set default figure placement to htbp
\makeatletter
\def\fps@figure{htbp}
\makeatother


% move the hyperref stuff down here, after header-includes, to allow for - \usepackage{hyperref}

\makeatletter
\@ifpackageloaded{hyperref}{}{%
\ifxetex
  \PassOptionsToPackage{hyphens}{url}\usepackage[setpagesize=false, % page size defined by xetex
              unicode=false, % unicode breaks when used with xetex
              xetex]{hyperref}
\else
  \PassOptionsToPackage{hyphens}{url}\usepackage[draft,unicode=true]{hyperref}
\fi
}

\@ifpackageloaded{color}{
    \PassOptionsToPackage{usenames,dvipsnames}{color}
}{%
    \usepackage[usenames,dvipsnames]{color}
}
\makeatother
\hypersetup{breaklinks=true,
            bookmarks=true,
            pdfauthor={Ross Dahlke
(\href{mailto:rdahlke@stanford.edu}{\nolinkurl{rdahlke@stanford.edu}}) ()},
             pdfkeywords = {},  
            pdftitle={Predicting Opinion Change in Deliberative Groups (Natural Language
Processing)},
            colorlinks=true,
            citecolor=blue,
            urlcolor=blue,
            linkcolor=magenta,
            pdfborder={0 0 0}}
\urlstyle{same}  % don't use monospace font for urls

% Add an option for endnotes. -----


% add tightlist ----------
\providecommand{\tightlist}{%
\setlength{\itemsep}{0pt}\setlength{\parskip}{0pt}}

% add some other packages ----------

% \usepackage{multicol}
% This should regulate where figures float
% See: https://tex.stackexchange.com/questions/2275/keeping-tables-figures-close-to-where-they-are-mentioned
\usepackage[section]{placeins}


\begin{document}
	
% \pagenumbering{arabic}% resets `page` counter to 1 
%
% \maketitle

{% \usefont{T1}{pnc}{m}{n}
\setlength{\parindent}{0pt}
\thispagestyle{plain}
{\fontsize{18}{20}\selectfont\raggedright 
\maketitle  % title \par  

}

{
   \vskip 13.5pt\relax \normalsize\fontsize{11}{12} 
\textbf{\authorfont Ross Dahlke
(\href{mailto:rdahlke@stanford.edu}{\nolinkurl{rdahlke@stanford.edu}})} \hskip 15pt \emph{\small }   

}

}






\vskip -8.5pt


 % removetitleabstract

\noindent \singlespacing 

\hypertarget{introduction}{%
\section{1. Introduction}\label{introduction}}

Deliberative Democracy is when a representative sample of people come
together for a short period of time to discuss civic issues. The idea is
that through a deliberative process, where people are access to
objective information and good conditions to discuss, people can come to
conclusions on civic issues that are most representative of what people
actually want {[}cite Fishkin book{]}. In a time where so much
discussion about political and civic issues happens in online
information bubbles, Deliberative Democracy offers a chance for people
to discuss issues on their merit and the freedom to change their
opinions in order to reveal a better representation of public opinion.

Deliberative Democracy Polling is a technique pioneered by Stanford
University Professor Jim Fishkin and the Center for Deliberative
Democracy (CDD). A Deliberative Poll is where individuals are polled on
their opinions before and after a weekend of small-group deliberation.
This method allows researchers to measure the change in people's
opinions as a result of the deliberative sessions.

This project uses the results of surveys and transcripts from the CDD's
Deliberative Polls to predict group opinion change based on the
transcripts from the groups' deliberative sessions. In talking with
members of the CDD, they theorized that computational approaches to
predicting opinion change might be difficult because of the narrow focus
of each of the deliberative sessions. For example, one whole session
might be devoted to discussing water policy. And so, water is the main
topic of discussion the entire time. Folks from the CDD have also
concluded that participants often don't explicitly say that they have
changed their opinion on the topic of the deliberative session, even if
they reveal a change in opinion in the survey. Given this prior
information, I will work on building a model to predict opinion change
in two steps:

\begin{enumerate}
\def\labelenumi{\arabic{enumi}.}
\tightlist
\item
  (For this milestone) Experimenting with predicting opinion change
  based on the words and text of the deliberative sessions.
\item
  (For the final submission) Experimenting with predicting opinion
  change by incorporating linguistic features
\end{enumerate}

Based on previous discussions with members of the CDC, I will first try
to find a baseline model based on the text found in the transcripts. For
the final project I will try to incorporate linguistic features into the
models. Incorporating linguistic features will be a key component of
this project because I hypothesize that it is not \emph{what} people
talk about in the deliberative sessions, it is \emph{how} they talk
about the issues. For example, how subjective or objective are the
statements people are making in the deliberative setting, or how are
sentences structured in groups that show opinion change versus don't
show opinion change?

\hypertarget{related-work}{%
\section{2. Related work}\label{related-work}}

\hypertarget{dataset-and-features}{%
\section{3. Dataset and Features}\label{dataset-and-features}}

\hypertarget{methods}{%
\section{4. Methods}\label{methods}}

\hypertarget{experiments-results-discussion}{%
\section{5. Experiments/ Results/
Discussion}\label{experiments-results-discussion}}

\hypertarget{next-steps}{%
\section{6. Next Steps}\label{next-steps}}

\hypertarget{references}{%
\section{References}\label{references}}





\newpage
\singlespacing 
\end{document}
